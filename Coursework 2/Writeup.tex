\documentclass[12pt, letterpaper]{article}
\usepackage{multicol}
\usepackage{natbib}
\usepackage{titling}
\usepackage{graphicx}
\graphicspath{{images/}}
\usepackage[toc,page]{appendix}
\usepackage{subcaption}
\setlength{\columnsep}{0.6cm}
\usepackage[margin=1.2in]{geometry}
\title{}
\author{William Ewart}
\date{March 2017}

\begin{document}
\begin{titlingpage}
\maketitle
\tableofcontents
\end{titlingpage}
\newpage

Investigate the effects of demographic factors on the accumulation or loss of cultural complexity
Henrich shows that with an increasing population size occasional inaccurate inferences (Individual errors during the transmission of a skill leading to learners gaining more skill) and selective choice of cultural model to copy will outweigh the degrading effect of low-fidelity transmission. Cause an increase in mean skill level in a population
Called cumulative adaptive evolution

Test how the number of sub-populations affect skill accumulation with varying levels of skill complexity

Split population density to high and low. Both densities had the same migratory activity. Varying skills complexities found that skill accumulation higher in the higher density population.

Was the same with varying migratory levels and skill complexities. The higher migratory levels caused more skill accumulation than the lower level.

More adults needed in lower population density at higher skill levels. Remains fairly consistent as the density increases with varying skill complexities.
In an isolated population the number of adults needed increases as the skill complexity increases.
Adult numbers increase with increasing skill complexity and increasing population density. At lower population densities the number of adults does not increase with increasing skill complexities.
\begin{multicols}{2}
\section{Introduction}
\section{Approach}
\section{Results}
\section{Conclusions}

\end{multicols}
\bibliographystyle{agsm}
\bibliography{bibliography}
\end{document}